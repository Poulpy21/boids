\documentclass[12pt,a4paper,sans]{article}

\RequirePackage[utf8]{inputenc}
\usepackage[french]{babel}
\usepackage{amsmath} 
\usepackage{amsfonts} 
\usepackage{amssymb} 
\usepackage{graphicx} 
\usepackage[margin=0.8in]{geometry}
\usepackage{stackrel}
\usepackage{float}

\hyphenpenalty=10000

\title{Simulation d'essaims particulaires}
\author{Keck Jean-Baptiste - Gauthier Zirnhelt}

\begin{document}
\maketitle

\section{Présentation du problème}
\paragraph{Étapes de la simulation}
% donner les étapes de la simu
\paragraph{Coûts des étapes de la simulation}
% surtout dans le calcul des forces (quadratique)

\section{Solution implémentée}
\subsection{Architecture générale}
% building the solution = cmake + sm_20 pour cuda
% différents éxecutables
% différents tests
% les arbres utilisés
\subsection{MPI}
% Architecture centralisée seulement pour le load balancing
\subsection{CUDA}
% mémoire coalescente
% optimisations diverses...
\subsection{Visualisation}
% Expliquer comment marche le viewer (merge fichiers, parsing fichier, sprites)

\section{Résultats}
% En gros, surtout des images commentées

% C'est pas mal ça
%\begin{minipage}{0.33\textwidth}
%	\begin{flushright}
%		\begin{figure}[H]
%			\centering
%			\includegraphics[width=\textwidth]{image.png}
%			\caption{Image}
%		\end{figure}
%	\end{flushright}
%\end{minipage}\\

\section{Analyse des résultats}
\subsection{Performances}
% Tableau comparatif : méthode(sequential, mpi seul, cuda seul, hybride) / nombre de boids (100-1000-10k-1M-...) [/ nombre de process / gpu]
\subsection{Qualité de la simulation}
% Vite dit, visuellement dans le viewer

\section{Perspectives d'amélioration}
% A voir à la fin

\end{document}
